\documentclass{article}
\pdfoutput=1 
% if you need to pass options to natbib, use, e.g.:
%     \PassOptionsToPackage{numbers, compress}{natbib}
% before loading neurips_data_2023

% ready for submission
\usepackage[preprint]{neurips_data_2023}
\usepackage{multicol}

% to compile a preprint version, add the [preprint] option, e.g.:
%     \usepackage[preprint]{neurips_data_2023}
% This will indicate that the work is currently under review.

% to compile a camera-ready version, add the [final] option, e.g.:
%     \usepackage[final]{neurips_data_2023}

% to avoid loading the natbib package, add option nonatbib:
%    \usepackage[nonatbib]{neurips_data_2023}

% Submissions to the datasets and benchmarks are typically non anonymous,
% but anonymous submissions are allowed. If you feel that you must submit 
% anonymously, you can compile an anonymous version by adding the [anonymous] 
% option, e.g.:
%     \usepackage[anonymous]{neurips_data_2023}
% This will hide all author names.

\usepackage[utf8]{inputenc} % allow utf-8 input
\usepackage[T1]{fontenc}    % use 8-bit T1 fonts
\usepackage{hyperref}       % hyperlinks
\usepackage{url}            % simple URL typesetting
\usepackage{booktabs}       % professional-quality tables
\usepackage{amsfonts}       % blackboard math symbols
\usepackage{nicefrac}       % compact symbols for 1/2, etc.
\usepackage{microtype}      % microtypography
\usepackage{xcolor}         % colors
\usepackage{caption}
\usepackage{subcaption}
\usepackage{natbib}
\usepackage{lipsum} 

% wrappable floats with text 
\usepackage{wrapfig} 

% Comment Commands
\newcommand{\ba}[1]{{{\textcolor{red}{[B: #1]}}}}
\newcommand{\sj}[1]{{{\textcolor{blue}{[S: #1]}}}}
\newcommand{\yy}[1]{{{\textcolor{green}{[Yu: #1]}}}}
\newcommand{\hy}[1]{{{\textcolor{purple}{[H: #1]}}}}
\newcommand{\yx}[1]{{{\textcolor{orange}{[X: #1]}}}}
\newcommand{\animals}{\textsc{SpuCoAnimals}\xspace}
\newcommand{\mnist}{\textsc{SpuCoMNIST}\xspace}
% For theorems and such
\usepackage{amsmath}
\usepackage{amssymb}
\usepackage{mathtools}
\usepackage{amsthm}
\usepackage{bbm}
\usepackage{bm}
\usepackage{enumitem}
\usepackage{comment}
\usepackage{pifont}
\usepackage{xspace}


% if you use cleveref..
\usepackage[capitalize,noabbrev]{cleveref}

\usepackage[export]{adjustbox}
\usepackage{array}
\usepackage{tabularx}
\usepackage{booktabs}
\usepackage{graphicx}

\def\tabularxcolumn#1{m{#1}}

%%%%%%%%%%%%%%%%%%%%%%%%%%%%%%%%
% THEOREMS
%%%%%%%%%%%%%%%%%%%%%%%%%%%%%%%%
\theoremstyle{plain}
\newtheorem{theorem}{Theorem}[section]
\newtheorem{proposition}[theorem]{Proposition}
\newtheorem{lemma}[theorem]{Lemma}
\newtheorem{corollary}[theorem]{Corollary}
\theoremstyle{definition}
\newtheorem{definition}[theorem]{Definition}
\newtheorem{assumption}[theorem]{Assumption}
\theoremstyle{remark}
\newtheorem{remark}[theorem]{Remark}

% Todonotes is useful during development; simply uncomment the next line
%    and comment out the line below the next line to turn off comments
%\usepackage[disable,textsize=tiny]{todonotes}
\usepackage[textsize=tiny]{todonotes}

\newcommand{\CE}{\text{CE}}

\newcommand{\x}{\pmb{x}}
\newcommand{\X}{\pmb{X}}
\newcommand{\y}{\pmb{y}}
\newcommand{\w}{\pmb{w}}
\newcommand{\rr}{\pmb{r}}
\newcommand{\vb}{\pmb{v}}
\newcommand{\D}{\mathcal{D}}
\newcommand{\OO}{\mathcal{O}}
\newcommand{\LL}{\mathcal{L}}
\newcommand{\J}{\mathcal{J}}
\newcommand{\N}{\mathcal{N}}
\newcommand{\A}{\mathcal{A}}
\newcommand{\B}{\mathcal{B}}
\newcommand{\ab}{\pmb{a}}
\newcommand{\bb}{\pmb{b}}
\newcommand{\NTK}{\pmb{\Theta}}
\newcommand{\bet}{\pmb{\beta}}
\newcommand{\mtx}{\bm} % bold matrix
\newcommand{\vct}{\bm} % bold vector
\newcommand{\mino}{\text{mino}}
\newcommand{\maj}{\text{maj}}
%%%%%%%%%%%%%%%%%%%%
\newcommand{\inner}[2]{\left\langle#1,#2\right\rangle}
\newcommand{\innerproduct}[2]{\langle #1, #2 \rangle}
\newcommand{\norm}[1]{ \left\| #1 \right\| }
%\newcommand\uniform{\overset{\text{unif.}}{\sim}}
%%%%%%%%%%%%%%%%%%%%
\DeclareMathOperator{\Tr}{Tr}
\DeclareMathOperator{\poly}{poly}
\DeclareMathOperator*{\E}{\mathbb{E}}
\DeclareMathOperator*{\argmax}{arg\,max}
\DeclareMathOperator*{\argmin}{arg\,min}
%%%%%%%%%%%%%%%%%%%%
\newcommand{\cmark}{\ding{51}}%
\newcommand{\xmark}{\ding{55}}%

\newcommand{\spuco}{\textsc{SpuCo}\xspace}
\newcommand{\DISPEL}{\textsc{Dispel}\xspace}
\newcommand{\SPARE}{\textsc{Spare}\xspace}

\title{Comparative Analysis of Contrastive Learning Methods Against Data Poisoning}

% The \author macro works with any number of authors. There are two commands
% used to separate the names and addresses of multiple authors: \And and \AND.
%
% Using \And between authors leaves it to LaTeX to determine where to break the
% lines. Using \AND forces a line break at that point. So, if LaTeX puts 3 of 4
% authors names on the first line, and the last on the second line, try using
% \AND instead of \And before the third author name.

\author{% 
  Mohammad Akbarnezhad \\
   \And
  Adithya Embar \\
   \And
    Dylan Gunn \\
   \And
  Anthony Holmes \\
  \And \vspace{-9mm}\\
  \\miles85@ucla.edu, aembar@ucla.edu, dylangunn@ucla.edu, anthonyjholmes1@ucla.edu\\
  Department of Computer Science, UCLA, Los Angeles, CA, 90024.\\
}


\begin{document}

\maketitle

\begin{abstract}

    \iffalse
\begin{enumerate}
    \item Summarize key objectives, methodology, findings, and significance.
    \item Keep it brief and enticing for readers to explore the full content.
\end{enumerate}
\fi

In recent times, the proliferation of data poisoning has emerged as a critical societal issue, casting a formidable shadow over the security of machine learning models. 
Over the past decade, we have witnessed a substantial surge in malicious poisoning attacks aimed at subverting the integrity of machine learning models.
In this paper, we shall embark upon an extensive examination of the repertoire of tools at our disposal for countering data poisoning attacks within the framework of Contrastive Learning with SAS. Our investigation will delve into the comparative effectiveness of these tools, providing insights into which among them yields the most robust defenses, all while elucidating the underlying rationales for their efficacy.

TODO: Add findings and significance.

Our code is available at : \url{https://github.com/anthonyjholmes1/com-sci-260D-Fall23}

\end{abstract}

\section{Introduction}

The advancements in contrastive learning techniques have propelled the capabilities of deep neural networks by fostering the extraction of rich and discriminative representations from unlabeled data. Despite their promise, the susceptibility of these models to poisoning attacks, particularly in the domain of contrastive learning, presents a formidable challenge. Poisoning attacks aim to corrupt the training data, introducing adversarial samples that can severely compromise the model's performance and reliability.

In the quest to fortify against these adversarial threats, this paper embarks on an exploration of the SAS (Subsets that maximize Augmentation Similarity to the full data) method as a core-set selection strategy within the realm of contrastive Semi-Supervised Learning (SSL). While SAS primarily aims to identify a subset of data points that alleviate computational burdens by maximizing similarity to the entire dataset under various augmentations, its performance under poisoned datasets in contrastive learning remains an uncharted territory.

The pivotal objective of this work is to evaluate the robustness and effectiveness of the SAS-selected core-set under the influence of poisoned data in contrastive SSL scenarios. Despite the valuable contributions of SAS in core-set selection for SSL, its behavior and adaptability when exposed to adversarial manipulations through poisoning attacks have yet to be thoroughly investigated in current research endeavors.

Through comprehensive empirical evaluations, we aim to shed light on the behavior of the SAS-selected subset when faced with poisoned datasets in contrastive learning environments. This investigation not only seeks to elucidate the response of the SAS method under adversarial scenarios but also endeavors to assess its resilience and suitability for practical deployment in scenarios susceptible to data poisoning.

In subsequent sections, this paper will delve into the fundamental principles of contrastive learning, outline the mechanisms behind poisoning attacks in this context, introduce the SAS core-set selection methodology, and present an extensive evaluation framework designed to examine the performance of SAS under the influence of poisoned datasets in contrastive SSL scenarios. The findings from this evaluation will contribute to a nuanced understanding of the robustness and effectiveness of the SAS method, addressing a critical gap in the current landscape of contrastive SSL research.

\section{Related Work}\label{sec:rel_work}


\paragraph{Contrastive Learning} 
Contrastive learning has emerged as a prominent paradigm in the domain of unsupervised representation learning, aiming to extract meaningful and discriminative features from raw data. The fundamental principle underlying contrastive learning revolves around the idea of learning representations by maximizing agreement between similar pairs of data samples while minimizing agreement between dissimilar pairs.


One of the foundational architectures in contrastive learning is the Siamese network, initially introduced for signature verification tasks. Siamese networks consist of twin networks sharing the same architecture and weights, where two input samples are processed through identical networks to generate embeddings. The objective of Siamese networks is to bring similar samples closer together in the embedding space while pushing dissimilar samples apart. This is typically achieved by utilizing contrastive loss functions, such as the contrastive loss introduced by Hadsell et al. in their seminal work~\cite{bertinetto2016fully}. The contrastive loss penalizes the model when embeddings of similar pairs are farther apart than a specified margin, and when embeddings of dissimilar pairs are closer than this margin.


Another influential development in contrastive learning is InfoNCE (Information Noise Contrastive Estimation), proposed by~\cite{oord2018representation}. InfoNCE is rooted in the framework of self-supervised learning, where the objective is to learn useful representations from unlabeled data. InfoNCE formulates the contrastive objective based on the mutual information between different views of the same instance. By maximizing the agreement between representations derived from different augmentations of the same data sample while minimizing the agreement between representations from different samples, InfoNCE facilitates the learning of robust and informative representations.


Recent advancements in contrastive learning have witnessed various extensions and improvements to the core principles. Approaches such as Momentum Contrast (MoCo)~\cite{he2020momentum}, SimCLR (Simple Contrastive Learning Representation)~\cite{chen2020simple}, and BYOL (Bootstrap Your Own Latent)~\cite{grill2020bootstrap} have demonstrated substantial performance gains by refining the mechanisms of generating positive and negative pairs, designing more effective augmentation strategies, or introducing novel pretext tasks.



\paragraph{Mechanisms Behind Poisoning Attacks in Contrastive Learning}

Contrastive learning models are vulnerable to poisoning attacks, where adversaries strategically inject malicious samples into the training data to manipulate the model's behavior. Understanding the mechanisms of these poisoning attacks is crucial in fortifying models against such adversarial threats. Adversarial perturbations play a pivotal role in poisoning attacks against contrastive learning models. Adversaries craft perturbations that are imperceptible to the human eye but can substantially alter the learned representations by the model \cite{goodfellow2014explaining}. These perturbations, when added to training data, aim to deceive the model during the learning process, causing it to misclassify or generalize poorly.

Poisoning attacks in contrastive learning can also manipulate the embedding space and decision boundaries of the model. By strategically placing poisoned instances in the training data, adversaries aim to shift the representations of specific classes or alter the decision boundaries, leading to misclassifications or biases in learned representations \cite{kim2020adversarial}. Another critical aspect is the transferability of poisoning attacks across different models or tasks. Adversarial examples generated to poison a specific contrastive learning model might generalize and cause disruptions in other models, making them susceptible to similar attacks \cite{papernot2016transferability}. Understanding this transferability is essential for devising robust defense strategies.

Poisoning attacks can significantly impact the learned representations and generalization capabilities of contrastive learning models. The introduction of poisoned instances during training can corrupt the learned representations, leading to compromised generalization performance and reduced model robustness against unseen data \cite{munoz2017towards}. Adversaries continually evolve attack strategies to evade existing defense mechanisms. Robust defense against poisoning attacks in contrastive learning requires a comprehensive understanding of these evolving adversarial techniques and the development of proactive defense strategies \cite{biggio2013evasion}.


\paragraph{SAS Core-Set Selection in Contrastive SSL}

Finding examples that significantly contribute to contrastive Semi-Supervised Learning (SSL) is notably challenging compared to core-set selection in supervised learning. Methods in supervised learning rely on loss or confidence of predictions, requiring labeled data. Contrastingly, SSL lacks labeled data, making it difficult to identify crucial examples.

SAS~\cite{pmlr-v202-joshi23b} addresses this challenge by maximizing alignment between augmented views within a class and minimizing dissimilarity between views of different classes. These examples, pivotal for contrastive SSL, pull together instances within a class and maintain learned class representation centers. Surprisingly,~\cite{pmlr-v202-joshi23b} observed that examples vital for contrastive SSL have less impact on supervised learning. Therefore, they concluded that high-confidence, low-forgetting-score examples can be safely excluded from supervised learning without affecting accuracy, while difficult-to-learn examples crucial in supervised learning might hinder contrastive SSL performance.

Extensive evaluations by~\cite{pmlr-v202-joshi23b} demonstrated SAS's efficacy across various datasets (e.g., CIFAR, STL10, TinyImageNet) and contrastive learning methods (e.g., SimCLR, BYOL). SAS subsets consistently outperform random subsets by over 3\% in downstream performance, efficiently extracting subsets critical for SSL early in training or using smaller proxy models. Nevertheless, a significant inquiry remains unaddressed is the performance assessment of the SAS-selected subset under a poisoned dataset is yet to be explored. This gap underscores the necessity to scrutinize the resilience and efficiency of the SAS-selected subset when exposed to poisoned data, representing a crucial aspect that has not been adequately investigated in current research.

\paragraph{Poisoning Data and Semi-Supervised Learning}

Resiliency tests carried out by poisoning the unlabeled data part of a dataset and training semi-supervised models on them found that a modification to the dataset as little as 0.1\% has the capacity to affect the classification power of the final model. In fact, Carlini claims that more accurate models are more vulnerable to poisoning attacks, dismissing the possibility that improving the models themselves could solve the underlying issue \cite{carlini2021}.

\section{Problem Formulation}

\begin{comment}
\begin{enumerate}
    \item Clearly define the problem or question your research aims to address.
    \item Try to be formal here (use mathematical notation as far as possible)
    \item Clearly state the specific setting you are exploring rather than the overarching problem
\end{enumerate}
\end{comment}

Clustering and classification methods, especially those employing supervised learning methods, are especially vulnerable to poisoning. By introducing a small subset of permuted or altered data, the accuracy of the entire resultant model trained on that data can be massively affected. Even on the scale of datasets used to train models that do work like spam identification in Gmail and malware detection in crowdsourced antivirus software, a targeted mass attack via submission of false negatives is enough to impact these models' classifications \cite{constantin2021data}.

Because contrastive learning methods make use of identified subsets that maximize expected augmentation similarity, the only way that a poisoned dataset can affect its performance is by affecting those learned minimal subsets; essentially, the permutations have to be drastic enough to affect the subsets that are chosen to maximize augmentation similarity in the feature space. The resultant subset is more eloquently described as follows:

$$S_k\;\; =\;\; arg min_{S \in V,|S| \leq r_k}\sum_{i\in V_k\S_k}\sum_{j\in S_k} d_{i,j},\quad d_{i,j} = <f(x_i), f(x_j)> $$

Theoretically, this should provide some robustness to dataset corruption and perturbation. While poisoning other types of data, like incorporating uncommon misspellings into a sentiment analysis-performing model, can cause massive reductions in performance, contrastive learning should not be so sensitive. For example, Google's toxicity detector can be nearly completely overridden by slightly changing some words; phrases that rank 90\% in "toxicity" easily are reduced to 10-15\% toxicity with the addition of basic perturbations \cite{hosseini2017}.

In our investigation, we aim to assess the robustness of contrastive learning frameworks, including notable methodologies such as MoCo, SimCLR, and CMC. To comprehensively evaluate their resilience against data poisoning, we will explore various key aspects identified in the literature. These critical facets encompass strategies such as Data Augmentation, Parallel Augmentation, Architectural choices, Loss Functions, and Data Modalities. For the scope of our study, we will exclusively focus on two widely utilized datasets, namely CIFAR-10 and CIFAR-10-P, both of which are image-based. As all of our experiments share the same data modality, the consideration of Data Modality variability becomes unnecessary in this particular context. CIFAR-10 was selected as our baseline dataset, providing a consistent reference point with the same model configuration. This choice allows us to establish a solid foundation for performance assessment, which we can then use to validate the outcomes when applied to the CIFAR-10P dataset.

Crucial to our research is the pursuit of a deeper understanding of the significance behind the choices made in data augmentation, parallel augmentation, architectural decisions, and loss function selection within the context of contrastive learning. We seek to elucidate why these choices have a bearing on the robustness of these frameworks.

Additionally, we will investigate whether there exists any discernible correlation between the choice of dataset and any of these key aspects. This examination will provide insights into whether certain datasets are more susceptible to the influence of these factors, shedding light on potential dataset-specific nuances that impact the performance of contrastive learning frameworks.


\section{Method}

\iffalse
\begin{enumerate}
    \item Describe the research design, approach, and methodology.
    \item Detail the procedures and tools used for data collection and analysis.
    \item Ensure clarity and reproducibility for readers.
\end{enumerate}
\fi

Our research approach involved an extensive review of existing literature in the field of contrastive learning. Given the evolving nature of this field, we observed a continuous progression of framework comparisons. To establish a fair basis for comparison, we selected frameworks that shared similarities and were released around the same time frame. This approach allowed each framework to showcase its strengths on a level playing field. Additionally, we conducted experiments spanning at least 50 epochs to ensure a robust evaluation of the models.

While pursuing our research objectives, we faced several challenges, primarily stemming from limitations in computational resources. Acquiring substantial computing power can be costly, which constrained the extent of our experiments. As a result, we limited our study to utilizing the CIFAR-10 and CIFAR-10C datasets, both widely recognized in the industry. It is worth noting that CIFAR-10C required manual download and incorporation into the PyTorch environment.Time constraints posed another challenge. Our goal was to train models with a broad range of parameters for extended durations to obtain comprehensive results. However, this approach had to be balanced against our available compute resources.

To ensure the reproducibility of our work, we have made our research repository openly accessible on GitHub \url{https://github.com/anthonyjholmes1/com-sci-260D-Fall23}. In this repository, we have provided links to the Google Colab files used to run our experiments. To facilitate comparison, we designed a single Google Colab file that contained all the different models. This setup allowed us to efficiently compare results, especially if a particular model achieved higher accuracy in fewer epochs. Detailed comments are provided at each stage in the Google Colab file, explaining our rationale for each component. Furthermore, we modularized the models into separate classes within the Colab file, enabling users to easily customize their experimentation by commenting out specific model sections.

\section{Experiments}

\begin{enumerate}
    \item Present the design and execution of your experiments.
    \item Report on the data collected and any statistical or analytical methods used.
    \item Provide sufficient detail for others to replicate your experiments.
\end{enumerate}

We conducted our experiments with 50 epochs that way we would be able to run numerous methods without worrying about the runtime of each execution. We kept the epochs the same to ensure consistency in our results. We trained and tested numerous contrastive learning approaches against clean and poisoned imaged based datasets found online(CIFAR10 and CIFAR10C). The methods we used are SimCLR, Moco and CMC. 

From our clean data set experiments we found that SimCLR had the highest accuracy of 92.8% out of all the methods. The next method with the highest accuracy was Moco with an 87.9% accuracy. The final method CMC was the least accurate at 83.7%.  

Compared to our poison datasets the results were very similar in that the methods were in the same order of most accurate to least. There were significant drop offs in accuracy % in each method howeever. SlimCLR was averaging 68.2%, Moco was averaging 65.7%, and CMC was averaging 61.2% accuracy. 

We have create a github repository that contains links to our google collab file that can be used to run the contrastive learning approaches. We have also included comments within each section as to explain what each part of the code blocks do. Within the collab file there are different code blocks that refer to a specific method that if a user wants to only run a specific on they can easily do so. The user can also change the number of epochs they want to run if they wanted to see how that effects the accuracy.


\section{Conclusion}\vspace{-2mm}

\begin{enumerate}
    \item Summarize the key findings and their implications.
    \item Reflect on how your results contribute to the broader context.
    \item Address any limitations and suggest avenues for future research.
\end{enumerate}

\newpage
\section{Contributions}

\paragraph{Mohammad Akbarnezhad}
\begin{enumerate}
    \item Created content for the presentation
    \item Wrote the report introduction
    \item Wrote the report related work
\end{enumerate}

\paragraph{Adithya Embar}
\begin{enumerate}
    \item Created content for the presentation
    \item Help present in presentation
\end{enumerate}

\paragraph{Dylan Gunn}
\begin{enumerate}
    \item Created content for the presentation
    \item Help present in presentation

\end{enumerate}

\paragraph{Anthony Holmes}
% Researched and read different papers. Created content and slides for the presentation. Presented in the presentation. Wrote the entire code for the project. In the report Anthony wrote the Abstract, Problem formulation, method, experiments, conclusion. 
\begin{enumerate}
    \item Created content for the presentation
    \item Help present in presentation
    \item Wrote the report abstract
    \item Wrote the report problem formulation
    \item Wrote the report methodology
    \item Wrote the project code
    
\end{enumerate}


%\section*{References}
\newpage
\bibliography{references}
\bibliographystyle{plainnat}

\end{document}
