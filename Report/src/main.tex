\documentclass{article}
\pdfoutput=1 
% if you need to pass options to natbib, use, e.g.:
%     \PassOptionsToPackage{numbers, compress}{natbib}
% before loading neurips_data_2023

% ready for submission
\usepackage[preprint]{neurips_data_2023}
\usepackage{multicol}

% to compile a preprint version, add the [preprint] option, e.g.:
%     \usepackage[preprint]{neurips_data_2023}
% This will indicate that the work is currently under review.

% to compile a camera-ready version, add the [final] option, e.g.:
%     \usepackage[final]{neurips_data_2023}

% to avoid loading the natbib package, add option nonatbib:
%    \usepackage[nonatbib]{neurips_data_2023}

% Submissions to the datasets and benchmarks are typically non anonymous,
% but anonymous submissions are allowed. If you feel that you must submit 
% anonymously, you can compile an anonymous version by adding the [anonymous] 
% option, e.g.:
%     \usepackage[anonymous]{neurips_data_2023}
% This will hide all author names.

\usepackage[utf8]{inputenc} % allow utf-8 input
\usepackage[T1]{fontenc}    % use 8-bit T1 fonts
\usepackage{hyperref}       % hyperlinks
\usepackage{url}            % simple URL typesetting
\usepackage{booktabs}       % professional-quality tables
\usepackage{amsfonts}       % blackboard math symbols
\usepackage{nicefrac}       % compact symbols for 1/2, etc.
\usepackage{microtype}      % microtypography
\usepackage{xcolor}         % colors
\usepackage{caption}
\usepackage{subcaption}
\usepackage{natbib}
\usepackage{lipsum} 

% wrappable floats with text 
\usepackage{wrapfig} 

% Comment Commands
\newcommand{\ba}[1]{{{\textcolor{red}{[B: #1]}}}}
\newcommand{\sj}[1]{{{\textcolor{blue}{[S: #1]}}}}
\newcommand{\yy}[1]{{{\textcolor{green}{[Yu: #1]}}}}
\newcommand{\hy}[1]{{{\textcolor{purple}{[H: #1]}}}}
\newcommand{\yx}[1]{{{\textcolor{orange}{[X: #1]}}}}
\newcommand{\animals}{\textsc{SpuCoAnimals}\xspace}
\newcommand{\mnist}{\textsc{SpuCoMNIST}\xspace}
% For theorems and such
\usepackage{amsmath}
\usepackage{amssymb}
\usepackage{mathtools}
\usepackage{amsthm}
\usepackage{bbm}
\usepackage{bm}
\usepackage{enumitem}
\usepackage{comment}
\usepackage{pifont}
\usepackage{xspace}


% if you use cleveref..
\usepackage[capitalize,noabbrev]{cleveref}

\usepackage[export]{adjustbox}
\usepackage{array}
\usepackage{tabularx}
\usepackage{booktabs}
\usepackage{graphicx}

\def\tabularxcolumn#1{m{#1}}

%%%%%%%%%%%%%%%%%%%%%%%%%%%%%%%%
% THEOREMS
%%%%%%%%%%%%%%%%%%%%%%%%%%%%%%%%
\theoremstyle{plain}
\newtheorem{theorem}{Theorem}[section]
\newtheorem{proposition}[theorem]{Proposition}
\newtheorem{lemma}[theorem]{Lemma}
\newtheorem{corollary}[theorem]{Corollary}
\theoremstyle{definition}
\newtheorem{definition}[theorem]{Definition}
\newtheorem{assumption}[theorem]{Assumption}
\theoremstyle{remark}
\newtheorem{remark}[theorem]{Remark}

% Todonotes is useful during development; simply uncomment the next line
%    and comment out the line below the next line to turn off comments
%\usepackage[disable,textsize=tiny]{todonotes}
\usepackage[textsize=tiny]{todonotes}

\newcommand{\CE}{\text{CE}}

\newcommand{\x}{\pmb{x}}
\newcommand{\X}{\pmb{X}}
\newcommand{\y}{\pmb{y}}
\newcommand{\w}{\pmb{w}}
\newcommand{\rr}{\pmb{r}}
\newcommand{\vb}{\pmb{v}}
\newcommand{\D}{\mathcal{D}}
\newcommand{\OO}{\mathcal{O}}
\newcommand{\LL}{\mathcal{L}}
\newcommand{\J}{\mathcal{J}}
\newcommand{\N}{\mathcal{N}}
\newcommand{\A}{\mathcal{A}}
\newcommand{\B}{\mathcal{B}}
\newcommand{\ab}{\pmb{a}}
\newcommand{\bb}{\pmb{b}}
\newcommand{\NTK}{\pmb{\Theta}}
\newcommand{\bet}{\pmb{\beta}}
\newcommand{\mtx}{\bm} % bold matrix
\newcommand{\vct}{\bm} % bold vector
\newcommand{\mino}{\text{mino}}
\newcommand{\maj}{\text{maj}}
%%%%%%%%%%%%%%%%%%%%
\newcommand{\inner}[2]{\left\langle#1,#2\right\rangle}
\newcommand{\innerproduct}[2]{\langle #1, #2 \rangle}
\newcommand{\norm}[1]{ \left\| #1 \right\| }
%\newcommand\uniform{\overset{\text{unif.}}{\sim}}
%%%%%%%%%%%%%%%%%%%%
\DeclareMathOperator{\Tr}{Tr}
\DeclareMathOperator{\poly}{poly}
\DeclareMathOperator*{\E}{\mathbb{E}}
\DeclareMathOperator*{\argmax}{arg\,max}
\DeclareMathOperator*{\argmin}{arg\,min}
%%%%%%%%%%%%%%%%%%%%
\newcommand{\cmark}{\ding{51}}%
\newcommand{\xmark}{\ding{55}}%

\newcommand{\spuco}{\textsc{SpuCo}\xspace}
\newcommand{\DISPEL}{\textsc{Dispel}\xspace}
\newcommand{\SPARE}{\textsc{Spare}\xspace}

\title{Comparative Analysis of Contrastive Learning Methods Against Data Poisoning}

% The \author macro works with any number of authors. There are two commands
% used to separate the names and addresses of multiple authors: \And and \AND.
%
% Using \And between authors leaves it to LaTeX to determine where to break the
% lines. Using \AND forces a line break at that point. So, if LaTeX puts 3 of 4
% authors names on the first line, and the last on the second line, try using
% \AND instead of \And before the third author name.

\author{% 
  Mohammad Akbarnezhad \\
   \And
  Adithya Embar \\
   \And
    Dylan Gunn \\
   \And
  Anthony Holmes \\
  \And \vspace{-9mm}\\
  \\miles85@ucla.edu, aembar@ucla.edu, dylangunn@ucla.edu, anthonyjholmes1@ucla.edu\\
  Department of Computer Science, UCLA, Los Angeles, CA, 90024.\\
}


\begin{document}

\maketitle

\begin{abstract}

    \iffalse
\begin{enumerate}
    \item Summarize key objectives, methodology, findings, and significance.
    \item Keep it brief and enticing for readers to explore the full content.
\end{enumerate}
\fi

In recent times, the proliferation of data poisoning has emerged as a critical societal issue, casting a formidable shadow over the security of machine learning models. 
Over the past decade, we have witnessed a substantial surge in malicious poisoning attacks aimed at subverting the integrity of machine learning models.
In this paper, we shall embark upon an extensive examination of the repertoire of tools at our disposal for countering data poisoning attacks within the framework of Contrastive Learning with SAS. Our investigation will delve into the comparative effectiveness of these tools, providing insights into which among them yields the most robust defenses, all while elucidating the underlying rationales for their efficacy.

TODO: Add findings and significance.

Our code is available at : \url{https://github.com/anthonyjholmes1/com-sci-260D-Fall23}

\end{abstract}

\section{Introduction}

\begin{enumerate}
    \item Set the context with background information.
    \item Outline the problem or question addressed.
    \item State the objectives and provide a roadmap for the study.
    \item Offer a more detailed narrative compared to the abstract.
\end{enumerate}

The advancements in contrastive learning techniques have propelled the capabilities of deep neural networks by fostering the extraction of rich and discriminative representations from unlabeled data. Despite their promise, the susceptibility of these models to poisoning attacks, particularly in the domain of contrastive learning, presents a formidable challenge. Poisoning attacks aim to corrupt the training data, introducing adversarial samples that can severely compromise the model's performance and reliability.

In the quest to fortify against these adversarial threats, this paper embarks on an exploration of the SAS (Subsets that maximize Augmentation Similarity to the full data) method as a core-set selection strategy within the realm of contrastive Semi-Supervised Learning (SSL). While SAS primarily aims to identify a subset of data points that alleviate computational burdens by maximizing similarity to the entire dataset under various augmentations, its performance under poisoned datasets in contrastive learning remains an uncharted territory.

The pivotal objective of this work is to evaluate the robustness and effectiveness of the SAS-selected core-set under the influence of poisoned data in contrastive SSL scenarios. Despite the valuable contributions of SAS in core-set selection for SSL, its behavior and adaptability when exposed to adversarial manipulations through poisoning attacks have yet to be thoroughly investigated in current research endeavors.

Through comprehensive empirical evaluations, we aim to shed light on the behavior of the SAS-selected subset when faced with poisoned datasets in contrastive learning environments. This investigation not only seeks to elucidate the response of the SAS method under adversarial scenarios but also endeavors to assess its resilience and suitability for practical deployment in scenarios susceptible to data poisoning.

In subsequent sections, this paper will delve into the fundamental principles of contrastive learning, outline the mechanisms behind poisoning attacks in this context, introduce the SAS core-set selection methodology, and present an extensive evaluation framework designed to examine the performance of SAS under the influence of poisoned datasets in contrastive SSL scenarios. The findings from this evaluation will contribute to a nuanced understanding of the robustness and effectiveness of the SAS method, addressing a critical gap in the current landscape of contrastive SSL research.


\section{Related Work}\label{sec:rel_work}

\begin{enumerate}
    \item Review existing literature relevant to your study.
    \item Highlight key findings and methodologies from previous research.
    \item Identify gaps or areas where your study contributes.
\end{enumerate}

\section{Problem Formulation}

\iffalse
\begin{enumerate}
    \item Clearly define the problem or question your research aims to address.
    \item Try to be formal here (use mathematical notation as far as possible)
    \item Clearly state the specific setting you are exploring rather than the overarching problem
\end{enumerate}
\fi

In our investigation, we aim to assess the robustness of contrastive learning frameworks, including notable methodologies such as MoCo, SimCLR, and CMC. To comprehensively evaluate their resilience against data poisoning, we will explore various key aspects identified in the literature. These critical facets encompass strategies such as Data Augmentation, Parallel Augmentation, Architectural choices, Loss Functions, and Data Modalities. For the scope of our study, we will exclusively focus on two widely utilized datasets, namely CIFAR-10 and CIFAR-10C, both of which are image-based. As all of our experiments share the same data modality, the consideration of Data Modality variability becomes unnecessary in this particular context. CIFAR-10 was selected as our baseline dataset, providing a consistent reference point with the same model configuration. This choice allows us to establish a solid foundation for performance assessment, which we can then use to validate the outcomes when applied to the corrupted CIFAR-10C dataset.

Crucial to our research is the pursuit of a deeper understanding of the significance behind the choices made in data augmentation, parallel augmentation, architectural decisions, and loss function selection within the context of contrastive learning. We seek to elucidate why these choices have a bearing on the robustness of these frameworks.

Additionally, we will investigate whether there exists any discernible correlation between the choice of dataset and any of these key aspects. This examination will provide insights into whether certain datasets are more susceptible to the influence of these factors, shedding light on potential dataset-specific nuances that impact the performance of contrastive learning frameworks.

\begin{table}[ht]
    \centering
    \caption{Comparing Models}
    \begin{tabular}{|c|c|c|c|}
        \hline
        Model & Parallel Augmentation & Architecture & Loss Function \\
        \hline
        MoCo V1  & Momentum Encoder & Momentum Contrast & InfoNCE \\
        \hline
        MoCo V2 & Momentum Encoder& Momentum Contrast& InfoNCE\\
        \hline
        SimCLR & Siamese Encoder & End to End & NT-Xent\\
        \hline
        CMC &  & Variance & InfoNCE\\
        \hline
    \end{tabular}
    \label{tab:example}
\end{table}



\section{Method}
\begin{comment}
\begin{enumerate}
    \item Describe the research design, approach, and methodology.
    \item Detail the procedures and tools used for data collection and analysis.
    \item Ensure clarity and reproducibility for readers.
\end{enumerate}
\end{comment}
To establish a fair basis for comparison, we made use of publicly-available MoCo and SimCLR models. We attempted to standardize our environment as much as possible to reduce the impact of confounding variables by only using implementations of these models in Torchvision; if the original papers' implementations were done on another framework, we only made use of alternative implementations that had proven equivalent performance on the same datasets, using accuracy as our performance measure, to the original models.

Each of the models was trained for 50 epochs on both CIFAR-10 and CIFAR-10-P datasets. It is important to note that the CIFAR-10-P dataset is a perturbed collection of the original CIFAR-10 dataset. While it is not optimally poisoned and is not targeted on any specific classes, it still functions similarly to an indiscriminate attack on a dataset, working to reduce the overall accuracy of the final model. This is done by applying transformations on the original images, including but not limited to: brightness changes, adding gaussian blur and noise, adding motion blur, layered effects, and more common and easy augmentations like rotation, scaling, and shearing of images.

The use of only 50 epochs for training each of these models may affect the final conclusions of our investigation. In many cases, standardized comparison of final accuracies between contrastive learning models is done on a scale of 200 to 500 epochs, at which point iterative gain is rather inconsequential and potentially causing the model to overfit to its training data. Our restriction on computational resources, being done on only a single GPU and in a time-constrained environment (Google Colab requires constant interaction, or will otherwise close the open instances), made it such that it was unreasonable to attempt anything more. This means that we will lack the tail behavior of training these models for more epochs, potentially causing us to make premature assumptions about the training behavior of these models for both unperturbed and perturbed data. It is also due to limited computational power that these models were trained on CIFAR-10 datasets, as the more commonly used ImageNet datasets are far too large and vastly reduce the maximum batch size that fits in our GPU memory.

To ensure the reproducibility of our work, we have made our research repository openly accessible on GitHub \url{https://github.com/anthonyjholmes1/com-sci-260D-Fall23}. In this repository, we have provided the Google Colab files used to run our experiments. Setup in individual Google Colab environments allowed easy parallelization of computation and comparison of results. All that is required for reproduction of our tests is for the individual Python notebook files to be imported into either a local or Colab instance and run. Our code and its specific imports rely on the existing Colab functionality; some packages may not be available on a local machine without additional installation.

All tests were run on instances containing single NVIDIA T4 GPUs, via Google Colab.

\section{Experiments}

\begin{enumerate}
    \item Present the design and execution of your experiments.
    \item Report on the data collected and any statistical or analytical methods used.
    \item Provide sufficient detail for others to replicate your experiments.
\end{enumerate}

\section{Conclusion}\vspace{-2mm}

\begin{enumerate}
    \item Summarize the key findings and their implications.
    \item Reflect on how your results contribute to the broader context.
    \item Address any limitations and suggest avenues for future research.
\end{enumerate}

\newpage
\section{Contributions}

\paragraph{Mohammad Akbarnezhad}
\begin{enumerate}
    \item Created content for the presentation
    \item Wrote the report introduction
    \item Wrote the report related work
\end{enumerate}

\paragraph{Adithya Embar}
\begin{enumerate}
    \item Created content for the presentation
    \item Help present in presentation
\end{enumerate}

\paragraph{Dylan Gunn}
\begin{enumerate}
    \item Created content for the presentation
    \item Help present in presentation

\end{enumerate}

\paragraph{Anthony Holmes}
% Researched and read different papers. Created content and slides for the presentation. Presented in the presentation. Wrote the entire code for the project. In the report Anthony wrote the Abstract, Problem formulation, method, experiments, conclusion. 
\begin{enumerate}
    \item Created content for the presentation
    \item Help present in presentation
    \item Wrote the report abstract
    \item Wrote the report problem formulation
    \item Wrote the report methodology
    \item Wrote the project code
    
\end{enumerate}


%\section*{References}
\newpage
\bibliography{references}
\bibliographystyle{plainnat}

\end{document}
