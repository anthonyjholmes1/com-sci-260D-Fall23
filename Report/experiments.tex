\section{Experiments}

\begin{enumerate}
    \item Present the design and execution of your experiments.
    \item Report on the data collected and any statistical or analytical methods used.
    \item Provide sufficient detail for others to replicate your experiments.
\end{enumerate}

We trained and tested two contrastive learning approaches against clean and perturbed imaged based datasets found online(CIFAR-10 and CIFAR-10-P). The contrastive learning methods we used are SimCLR and Moco. With the help of online implementations we had used certain code chunks to formulate our experiments while excluding extraneous code that introduced new variables to consider. We conducted our experiments with 50 epochs that way we would be able to run numerous methods without worrying about the runtime of each execution. We felt this was the appropriate number as we were able to see  distinctions in the accuracy while also being able to run numerous times which allowed us to take an average when comparing our results. We also kept the epochs the same for both methods to ensure consistency in our results. 

From our clean data set experiments we found that SimCLR had the higher accuracy of 92.8 compared to the Moco implementation with an 87.9 accuracy. 

Compared to our perturbed datasets the results were very similar in that the methods were in the same order of most accurate to least. There were significant drop offs in accuracy percentage in each method however. SlimCLR was averaging 68.2, and Moco was averaging 65.7. 

We have create a github repository that contains links to our google collab file that can be used to run the contrastive learning approaches. We have also included comments within each section as to explain what each part of the code blocks do. Within the collab file there are different code blocks that refer to a specific method that if a user wants to only run a specific on they can easily do so. The user can also change the number of epochs they want to run if they wanted to see how that effects the accuracy.
